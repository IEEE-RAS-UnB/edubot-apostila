\addcontentsline{toc}{chapter}{Introdução}
\chapter*{Introdução}
\justifying
\paragraph{}
Essa apostila foi criada pela equipe da \textbf{EDUBOT} para te ajudar a desvendar o \textit{Sparki}, esse robozinho simpático muito bem equipado, produzido pela \textbf{Arcbotics}. E, com isso, mostrar que programar todos sensores, motores, tela, luzes e acessórios que ele possui pode ser bem simples e divertido, e não um monstro de sete cabeças como a maioria das pessoas pensa.
\paragraph{}
Sendo assim, buscamos usar também uma linguagem mais simples e inserir diálogos em forma de perguntas e repostas que, por ser tão comum em salas de aula, vocês já devem estar familiarizados.
\paragraph{}
Nossa principal fonte para elaboração desse material didático é o site da própria Arcbotics e se você quiser entrar para dar uma conferida, vai ver até imagens semelhantes, pois tiramos muitas imagens de lá. É por lá também que adquirimos os robôs. \\
Já deixo aqui o link para facilitar: \url{http://arcbotics.com/}
\paragraph{}
O \textit{Sparki} vem pronto, com sensor de distância e de luz, acelerador, tela LCD, LED RGB, campainha, controle remoto, dois motores e a mesma IDE do Arduino. Nessa apostila, vamos aprender a programar tudo isso, além de poder ver na prática alguns assuntos que vemos em Matemática e Física durante o Ensino Médio.
\paragraph{}
Além disso, aprender a programar pode ser bem útil na sua vida pessoal, dentre as vantagens, cito aqui algumas:  o pensamento computacional traz uma grande metodologia pra solucionar problemas, partindo de questões menores para depois resolver o todo, esse mesmo pensamento computacional pode melhorar até sua organização pessoal, além de desenvolver seu raciocínio lógico, aumentando a  clareza, a rapidez e a fluidez de seus pensamentos. Ah! Também será ótimo para aprender um pouco mais de inglês.
\paragraph{Quem somos nós?} Acreditando na diferença que a programação e a robótica podem fazer na trajetória escolar e profissional de uma pessoa, a EDUBOT é um projeto de extensão da Universidade de Brasília (UnB) e um projeto apoiado pelo capítulo estudantil de Robótica e Automação (RAS) do ramo estudantil da IEEE (Institute of Electrical and Electronic Engineers) na UnB, que tem como principal objetivo o incentivo à formação de jovens nas áreas de Tecnologia e Engenharia através da realização de oficinas de robótica como atividades extracurriculares em escolas e instituições públicas de ensino de nível médio do Distrito Federal. \\
Para saber mais, acesse: \\
\url{https://www.facebook.com/projetoedubot/} \\
\url{ https://site.ieee.org/sb-unb/capitulos-estudantis/ras_unb/}\\
\url{https://www.ieee-ras.org/} 
\paragraph{E aí, pronto para começar?}


